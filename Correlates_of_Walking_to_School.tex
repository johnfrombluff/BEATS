---
title: "Correlates of Walking to School"
author: "John Williams"
bibliography: /home/john/Dropbox/Writing/bib/all-refs.bib
csl: /home/john/Dropbox/Writing/bib/CSL/apa.csl
output:
  Grmd::docx_document:
    fig_caption: TRUE
    force_captions: TRUE
    toc: TRUE
    toc_depth: 5
  pdf_document:
    toc: true
    latex_engine: xelatex
    keep_tex: true
---

```{r label=R-setup, echo=FALSE, include=FALSE, cache=FALSE}
# bibliography: /home/john/Dropbox/Writing/bib/all-refs.bib
require( ggplot2 )
require( scales )
require( Gmisc )
require( rms )
require( knitr )
require(JMisc)
# Evaluate the figure caption after the plot
knitr::opts_knit$set(eval.after='fig.cap')

# Use the table counter that the htmlTable() provides
options(table_counter = TRUE)

# Use the figCapNo() with roman letters
options(fig_caption_no_roman = TRUE)

#theme_set( theme_gray( base_size = 10 ))
#theme_update( legend.key.width = unit( 3,"line") )
options(width=120)
options("show.signif.stars"=F)

## set global chunk options
opts_chunk$set(echo=FALSE,
               cache=TRUE,
               dpi=96,
               fig.height=5,
               fig.width=7,
               prompt=F,
               tidy=T,
               highlight=T,
               dev="png",
               dev.args=list(type="cairo"),
               fig.align='center',
               fig.show='hold',
               par=TRUE,
               comment=NA,
               background="wheat",
               prompt=FALSE,
               warning=FALSE,
               message=FALSE)

info   <- sessionInfo()
r_ver  <- paste(info$R.version$major, info$R.version$minor, sep=".")
barcol <- "wheat"
source("/home/john/Dropbox/Writing/Software/R/JMisc.R")
source("/home/john/Dropbox/Writing/Software/R/logistic-regression-functions.R")
```

# Notes to the research team
This file contains two things:

1. A record of the analyses that I've done, with full details of the statistical
results.  More details can be shown on request
2. Snippets that can be copied and pasted into manuscripts.  (Note the abstract
is at the end because the numbers in the abstract rely on code occurs later in
the file.)

All analyses were performed using **R**, ver. `r r_ver` [@R15] with packages
rms, ver. `r info$otherPkgs$rms$Version` [@rms15] for  analysis;
Gmisc, ver. `r info$otherPkgs$Gmisc$Version` for plot and table output; and
knitr, ver `r info$otherPkgs$knitr$Version` [@xie14] for reproducible research.

## Reproducible research
This file is automatically generated, i.e. all the in-text numbers, tables and
graphs are generated from the data.  If the data file changes, or we decide to
analyse a different subset of the data, or include or exclude certain variables,
the content in  the document will change automatically (almost), i.e. without
the need to re-type anything.

"Weaving" text, code and results together and rendering them into a document is
known as "reproducible research", i.e. there is code in the source file that
reads the data and performs the analysis, then generates the document. This
makes your research report more tightly bound to the data, and makes it less
likely that you will be unable to reproduce or extend your research in the
future if the need arises (as it often does, in my experience). You can open
the file with the same name as this one, but the extension ".Rmd"
to see the  source file which is compiled into HTML by **R**.

This approach can
also generate PDF (which some journals accept), but as yet there is no one-step
method for rendering directly to an editable format, e.g. odt, rtf or docx. But
it's easy to simply copy and paste from your web-browser into your
word-processor. But of course you should not do this until everyone involved
with the manuscript preparation is happy that the results are 100% finalised!

## Data Setup
The **R** code below is just to show which data file is being used.

```{r label="read-data", echo=2:4, size="small", prompt=FALSE}
require(foreign)
dir  <- "/home/john/Dropbox/Research/Collaboration/BEATS/John/W2S"
file <- "BEATS_SS_ForWalk2School_150507.sav"
dat <- read.spss( paste(dir, file, sep="/"), to.data.frame=TRUE )
rm(dir, file)
```


```{r label=data-setup, results='hide', cache=TRUE, dependson="read-data"}
source( "data-setup-W2S.R")
```

# Introduction

We know from previous work (and common sense!) that the most influential
correlate is distance from school. But once that is factored out, what else is
influential?

```{r label=ATS-Dist, dependson="data-setup", fig.cap="Figure 1: Empirical probability of ATS by distance from school", fig.align='center'}
emp <- NULL
plot_cutoff <- 10000
with( dat.full[dat.full$Dist2School <= plot_cutoff, ],
      {
        for ( i in 1:length(Dist2School ) )
           emp[i] <- prop.table(table(ATS_f[ Dist2School < Dist2School[i]]))["Walk"]
        emp_prob <- data.frame( dist=Dist2School, true=ATS_f, prob=emp )
        p <- ggplot( emp_prob, aes( dist, prob) )
        p <- p + labs(x="Distance to school (m)", y="Probability of Walking to School" )
        p <- p + geom_point(size=1)
        p <- p + geom_smooth( size=1, method="gam", formula= y ~ s( x, bs = "cs") )
        p <- p + annotate("text", label="Optimal distance\nthreshold: ≤ 2,200m\nSensitivity: 84%\nSpecificity: 84%\nAUC: 93%", x=4000, y=0.8, hjust="l")
        p + geom_vline(x=2200)
        }
      )
```

The figure above shows the empirical probability of walking to school, i.e. the
proportion of respondents walk at each level of distance to school. The
observations plotted are those respondents who live less than
`r I(format(plot_cutoff, big.mark=","))`m from school to make the plot more interpretable. The blue
line is a Generalised Additive Model smoother.

```{r walkstats, results='asis', cache=FALSE}
source( "data-setup.R")
walk_min <- 469
walk_50  <- 3100
walk_max <- 5800

tmp <- dat$ATS_f[dat$Dist2School <= walk_min]
walk_min_n <- sum(table(tmp))

tmp <- dat$ATS_f[dat$Dist2School <= walk_50  ]
walk_50_n  <- sum(table(tmp))

tmp <- dat$ATS_f[dat$Dist2School > walk_max  ]
walk_max_n <- sum(table(tmp))
```

Of the  `r I(walk_min_n)` students who live less than `r I(walk_min)`m from
school, all  walk to school. Of the `r I(walk_50_n)` who live less than
`r I(walk_50)`m, 50% walk.  Finally, of the `r I(walk_max_n)` students who live
more than `r I(walk_max)`m from school, none of them walk to school.

# Sample description

## Exclusion of cases and missing value analysis
The analyses below are restricted to the students who are not boarders. There are
`r I(format(dim(dat.ats)[ 1 ], big.mark=","))` students who fit those criteria.
The table below shows the number of missing values on variables to be included
in the multivariate  analyses, which reduce the available sample size.

<center>
```{r mva, dependson="data-setup", cache=FALSE}
# 2 obs  (9078, 9038) had misgin aGender but non-missing gender; and missing
# age_cat but non-missing school year, DOB_day, DOB yeaer ats. I entered 12.5
# for Age_Cat
# 4 (9165, 4002, 1034, 10152) had missing values for all WSC sections etc.
# Maybe they didn't finish the questionnaire?
# 11 (9165, 4002, 1034, 10152, 9104, 3027, 9010, 8141, 8175, 3087, 5108) have
# missing value on W2S section
mva( dat.ats, thr=4 )
```
</center>

The variables with 11 cases missing are due to participants not completing a
section of the questionnaire (due to time constraints?).  Not all students
consented to, or had time available for, anthropometry, so many of these values
are missing. It's not clear to me why there are so many missing values for the
other variables though.

A total of `r I(fmt(dim(dat.ats)[ 1 ]))` adolescents were
included in the analysis (Age
`r I(round(mean(dat.ats$Age_Cat, na.rm=T),1))` ±
`r I(round(sd(dat.ats$Age_Cat, na.rm=T),1))` years;
`r I(round(100*prop.table(table(dat.ats$gender)),1)["Male"])`% boys;
`r I(round(100*prop.table(table(dat.ats$eth3)),1)["NZ European"])`% New Zealand European;
`r I(round(100*prop.table(table(dat.ats$BMI_f)),1)["Normal"])`% normal weight).
The most common modes of transport to school was being driven by others
(`r I(sum(round(100*prop.table(table(dat.ats$TscCarOth)),1)[4:5]))`%)
followed by walking
(`r I(sum(round(100*prop.table(table(dat.ats$TscWalk)),1)[4:5]) )`%),
school bus (`r I(sum(round(100*prop.table(table(dat.ats$TscBusSc)),1)[4:5]))`%),
public bus (`r I(sum(round(100*prop.table(table(dat.ats$TscBusPub)),1)[4:5]))`%)
and driving themselves
(`r I(sum(round(100*prop.table(table(dat.ats$TscCarMy)),1)[4:5]))`%).
Overall,
`r I(round(100*prop.table(table(dat.ats$ATS)),1)[1])`%
of adolescents used motorized transport only,
`r I(round(100*prop.table(table(dat.ats$ATS)),1)[2])`%
W2S, and
`r I(round(100*prop.table(table(dat.ats$ATS)),1)[3])`%
used a combination of motorized and active transport to school. Most students
(`r I(round(100*prop.table(table(dat.ats$TSlike)),1)["Yes"])`%)
liked the way they travelled to school.


## Potential correlates
Socio-demographic characteristics of students who walked versus did not walk to
school are presented in Tables 2, 3 and 4.

<center>
```{r label="descriptives-demos",cache=FALSE}
require( Gmisc )
source( "data-setup.R" )
n.w  <- table(dat.ats$ATS_f)[2]
n.dw <- table(dat.ats$ATS_f)[1]
res  <- data_setup( dat.ats,  "ATS_f", c("Age_Cat", "gender", "eth3",
                                         "BMI_4cat", "cars3", "NZDepCat3",
                                         "tsdecision","Dist2School"))
(tab1 <- htmlTable(x = res$tab,
  rgroup   = res$rgroup,
  n.rgroup = res$ngroup,
  label    = "Table1",
  caption  = "Demographic characteristics of the sample.",
  tfoot    = "<small><sup>&dagger;</sup>Categorical variables are reported in counts and percentages: count (%). The <i>p</i>-values are from Fisher tests. Proportions are calculated horizontally. Continuous variables are reported as mean (±SD). The <i>p</i>-values are from Wilcoxon tests.</small>",
  rowlabel = "Variable<sup>&dagger;</sup>",
  css.rgroup = "font-weight: 100",
  #cgroup   = c(n.dw,             n.w, ""),
  #n.cgroup = c(1,                1, 1 ),
  ctable   = TRUE ))
```
</center>

The tables below are not intended for inclusion in manuscripts, they are just
here for reference in case we decide to change anything else.

<center>
```{r label="descriptives-cat", cache=FALSE}
require( Gmisc )
source( "data-setup.R" )
res <- data_setup( dat.ats, "ATS_f", c("school", "BMI_f", "BMI_2cat","HMcars", "ScrGuide",
                                       "whodecides", "schiclose"))
(tab1 <- htmlTable(x = res$tab,
  rgroup   = res$rgroup,
  n.rgroup = res$ngroup,
  label    = "Table1",
  caption  = "Categorical individual and household potential correlates of walking to school",
  tfoot    = "<small><sup>&dagger;</sup>Variables are reported in counts and percentages: count (%). The <i>p</i>-values are from Fisher tests. The total proportions are calculated vertically, and the others are calulated horizontally.</small>",
  rowlabel = "Variable<sup>&dagger;</sup>",
  css.rgroup = "font-weight: 100",
  ctable   = TRUE ))
rm( tab1 )
```
</center>

&nbsp;

<center>
```{r descriptives-cont, cache=FALSE}
require( Gmisc )
source( "data-setup.R" )
res <- data_setup( dat.ats , "ATS_f", names(dat.ats)[c(16, 13:14, 17:50)])
(tab2 <- htmlTable(x = res$tab,
  rgroup   = res$rgroup,
  n.rgroup = res$ngroup,
  label    = "Table1",
  caption  = "Continuous individual potential correlates of walking to school",
  tfoot    = "<small><sup>&dagger;</sup>The <i>p</i>-values are from  Wilcoxon tests.</small>",
  rowlabel = "Variable<sup>&dagger;</sup>",
  css.rgroup = "font-weight: 100",
  ctable   = TRUE ))
rm( tab2 )
```
</center>

# Modeling
Notes:

- The effect of distance on the probability of walking to school is clearly
non-linear, so the distance variable was transformed using a restricted cubic
spline (with three knots).
- Because the data were collected within schools, robust standard errors were
calculated using school as a cluster variable.
- Following @hosmer13 [p. 177], models with areas under the  ROC curve greater
than 0.9 are labelled "outstanding", and those with ROC areas between 0.8 and
0.9 are labelled "excellent".

## Model 1
Following best practice (not *common* practice), as explained by
@harrell01 [pp. 56--60], all significant univariate correlates from Table 1 were
included in the initial model. This model was reduced by removing the correlate
with the largest *p*-value one at a time and re-inspecting the fit carefully.

This was not a purely automatic or data-driven process. At each stage, the
conceptual meaning of the candidate variable for removal was considered.  During
this process, it became apparent that two of the correlates, **time** and
**intention** are highly correlated with distance and with each other, even though
their VIFs are well below 10. Also these variables have little explanatory power
conceptually. Accordingly they were removed at the earliest stage of model
simplification.

```{r ATS-m1, dependson=c("read-data"), results='hide'}
source("data-setup.R")
options(contrasts=c("contr.treatment", "contr.treatment"))
dat.m1   <- dat.ats[ complete.cases(dat.ats), ]
m1.ddist <- datadist(dat.m1)
options(datadist="m1.ddist")
m1 <- robcov(lrm(ATS_f ~ rcs( Dist2School, knots=3)
          + BMI_f + n_cars + NZDepCat3 + school_decile_n
          + whodecides + control + schiclose + closest
          + interesting + pleasant + boring + useful + safe + exercise
          + onway + stuff + sched + planning + sweaty + unsafe + tired
          + desire + confd
          + adults + parents_walk + parents_safe
          + parents_say + friends_say + school_says + cool + friends_dont
          + weather + hills + regwalk
          + NEStConnect + NGEsthetics,
          x=T, y=T, data=dat.m1 ), dat.m1$school)
m1.gof <- LR.summary( m1, html = T, anova=F )
```

<center>
```{r m1_gof, cache=FALSE}
htmlTable(m1.gof$sf,
          rnames=F,
          align="r",
          header=c("Index", "<span style='padding:5mm'/>", "Statistic", "<span style='padding:5mm'/>"),
          caption="Inital model goodness of fit and summary information")
```
</center>


<center>
```{r m1_anova, results='asis'}
#aov2html( m1 )
```
</center>

## Model 2
The variables that survived this process and remained significant at the 5%
level are shown below, in model 2. Although the goal was to reduce the set of
correlates to only those significant at the 5% level, meeting that criteria
resulted in a set which was also significant at the 1% level.

```{r ATS-m2, dependson=c("data-setup", "ATS-m1"), results='asis'}
attach(dat.ats)
dat.m2 <- dat.ats[ complete.cases(ATS_f, Dist2School,
                                  BMI_2cat,
                                  onway, sched, planning,
                                  confd, parents_say, cool, regwalk),
                   c("ID", "ATS_f", "Dist2School",
                     "BMI_2cat",
                     "onway", "sched", "planning",
                     "confd", "parents_say", "cool", "regwalk",
                     "school") ]
detach( dat.ats )
sc <- dat.m2$school
m2.ddist <- datadist( dat.m2 )
options( datadist = "m2.ddist", contrasts=c("contr.treatment", "contr.treatment") )
m2 <- robcov(lrm(ATS_f ~ rcs(Dist2School, 3) + BMI_2cat
          + onway + sched + planning
          + parents_say + cool + regwalk,
          x=T, y=T, data=dat.m2 ), cluster=sc )
m2.gof <- LR.summary( m2, print=F )
```

<center>
```{r m_final_gof, cache=FALSE}
htmlTable(m2.gof$sf,
          rnames=F,
          align="r",
          header=c("Index", "<span style='padding:5mm'/>", "Statistic", "<span style='padding:5mm'/>"),
          caption="Goodness of fit and summary information")
```
</center>

Table `r I(options("table_counter"))` summarises the goodness of fit of Model 2.

<center>
```{r m_final_anova, results='asis', cache=FALSE}
aov2html( m2 )
```
</center>

Table `r I(options("table_counter"))`  shows Wald tests of the covariates. All
correlates are significant at the 1% level. The 5% level leaves to much room
for Type I error due to sample size and alpha inflation, so the 1% level was
used to avoid these problems.  Regardless of *common practice* this is a wise
strategy to avoid false positives and non-replicable scientific research
[@ioannidis05].


```{r ATS-final, dependson=c("ATS-m2", "data-setup"), results='asis'}
m_final <- m2
m_final_gof <- LR.summary( m_final, print=FALSE )
#htmlTable(round(anova(m_final), 3))
```

To address the possibility of over-fitting, bias-corrected goodness of fit
indices were calculated using `r validate_n<-20; I(validate_n)` bootstrap
samples. (**NB**: currently this is set rather low, so that this file will
compile  quickly. I will update it to be larger if required, but I've previously
inspected the calibration results with *n*=200, giving the same substantive
conclusion.)

<center>
```{r label=ATS-final-validate, dependson="ATS-final",results='asis', cache=F}
s <- validate( m_final, B=validate_n )
colnames(s) <- c("Model", "Training", "Test", "Optimism", "Corrected", "n")
rownames(s)[1] <- "R<sub>xy</sub>"
rownames(s)[2] <- "R<sup>2</sup>"
htmlTable( round(s, 3),
           caption="Model bootstrap calibration",
           rowlabel="Index",
           align="r",
           label="m_final_calibration")
```
</center>

Table `r I(options("table_counter"))` shows that the bias-corrected indices  are
very similar to the model indices, hence the results are unlikely to be due to
over-fitting, and are more likely to be generalisable to new samples from the
same population.

<center>
```{r m2-results-table, cache=FALSE, results='asis', dependson="ATS-final"}
#LR.summary(m_final, table=T, print=F)
htmlTable( OR(m2, Dist2School=c(4000,5000)),
          label="OR-M2",
          caption="Odds ratios<sup>&dagger;</sup>",
          align="r",
          tfoot="<small><sup>&dagger;</sup> Odds ratios are functions of the difference column, not the usual 1-unit calculation. The low and high values are the IQRs (except for distance, where the low and high values were chosen for clarity of interpretation).</small>")
```
</center>

Table `r I(options("table_counter"))` shows the odds ratios from Model 2 and
Figure 2 below represents the odds ratios and their confidence intervals
graphically.

```{r ATS-OR-plot, dependson="ATS-final",fig.width=7.5, fig.cap="Figure 2: Odds ratios"}
plot(summary(m_final, Dist2School=c(4000,5000)))
```

Figure 3 below shows how the log odds of W2S varies over the range of the
correlates in the final model, with shaded areas indicating 95% confidence
intervals.

```{r ATS-OR-logit-plot, dependson="ATS-final",fig.width=8.5, fig.cap="Figure 3: Effect of covariates on logit"}
options(datadist="m1.ddist")
res <- Predict(m_final)
#plot(res[res$.predictor != "Dist2School", ])
plot(res)
m4.gof <- LR.summary(update(m_final, . ~ . - rcs(Dist2School,3)), print=F )
```


## Interpretation
The final model has absolutely outstanding predictive and discriminant validity,
and all  effects can be interpreted *ceteris paribus*. It also has good face
validity. Apart from the obvious effect of distance
(`r I(OR.inline(OR(m_final, Dist2School=c(4000,5000))["Dist2School",], 3))`),
logistical factors:

- school being on the way to somewhere else,(**onway**, OR `r I(OR.print(m_final, "onway", 2))`) most probably work, or the school of a sibling) and

- the difficulty of fitting walking into the student's after-school schedule (**sched**, OR `r I(OR.print(m_final, "sched", 2))`)

- planning for walking to school (**planning**, OR `r I(OR.print(m_final, "planning", 2))`)

decrease the likelihood of walking to school. However social and lifestyle factors:

- peer approval (**cool**, OR `r I(OR.print(m_final, "cool", 2))`),

- having parental support or encouragement to walk (**parents_say**, OR `r I(OR.print(m_final, "parents_say", 2))`)

- being a walker for general mobility (**regwalk**, OR `r I(OR.print(m_final, "regwalk", 2))`), and

increase the likelihood of W2S.  Note that walking for general mobility may be
a lifestyle choice, or a function of lack of alternative mobility options.
However the correlation between **regwalk** and **n_cars** is only
`r I(with(dat.ats, myCor(n_cars, regwalk)))`
(*p* &lt; `r I(fmt.p(with(dat.ats, cor.test(n_cars, regwalk))$p.value))`),
which gives some evidence that regular walking may not be based largely or
solely on necessity.

Lastly, BMI has an effect.  The point estimates for ORs of being overweight or
obese more than halve the odds of W2S. The normal vs. obese comparison is not
significant, but this is almost certainly due to the lower number of students in
this category inflating the standard error, as the graphical presentation of ORs
shows.

The final model suggests that parental factors of encouragement to walk to
school, and to walk for general mobility, will be the most influential factors in
promoting walking to school among New Zealand secondary school students. If this
encouragement was genuine, then the barriers of convenience and after-school
schedule would not apply: students could be dropped off somewhere other than the
school gate in the morning, and picked up from the school in the afternoon.
Then, at least half the time they would be walking.

It is relevant to note that there were many other "healthy lifestyle" factors
that were unrelated to walking to school. No variables in the nutrition or
physical activity sections of the questionnaire influence the likelihood of
walking, for example.

As a final comment, one may think that because distance has such a large
influence on walking to school, that the additional variables in the final model
do not have much explanatory power in relative terms. However this is not true.
Removing distance to school from the final model still gives an ROC value of
`r I(m4.gof$stats$ROC)`%, which is still outstanding.

# Summary
All logistic regression modelling results are exceptionally good in terms of
discrimination, predictive ability and face validity.

But, if time permits, I would really like to re-run these  models excluding
distances where either everyone walks (about 500m) or no-one  walks (about
5000m).  That would reduce the sample size somewhat, but I feel it would also
increase both the substantive and statistical validity and robustness of
modelling results.

# Abstract
**Results**: Overall,
`r I(round(100*prop.table(table(dat.ats$ATS)),1)[1])`%
of adolescents in the sample used motorized transport to school,
`r I(round(100*prop.table(table(dat.ats$ATS)),1)[2])`%
W2S, and
`r I(round(100*prop.table(table(dat.ats$ATS)),1)[3])`%
used both motorized and active transport to school. Univariate correlates of W2S
included perceptions of walking to school, benefits of exercise, socializing with
friends, perceived barriers (convenience of being driven, time constraints,
school bag weight, after-school schedule, planning, sweating, safety concerns,
being too tired, lack of interest/desire), perceived control, number of adults
and vehicles at home, encouragement (peers, parents, school), distance,
perceived neighbourhood environment (land-mix use access, street connectivity,
aesthetics, hills) and weather. In a multivariate model,
peer approval (OR [95% CI]:
(`r I(OR.print(m_final, "cool", 2)) `) and parental encouragement
(`r I(OR.print(m_final, "parents_say", 2))`)
were positively associated with W2S while incompatibility of walking with
after-school schedule
(`r I( OR.print(m_final, "sched", 2))`)
and convenience of being driven to school on the way to somewhere else
(`r I(OR.print(m_final, "onway", 2) )`)
were negative factors.

# References
